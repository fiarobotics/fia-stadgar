% -*- mode: LaTeX; mode: flyspell; mode: auto-fill; -*-
\documentclass[11pt,a4paper]{article}

\usepackage[swedish]{babel}
\usepackage[utf8]{inputenc}
\usepackage[T1]{fontenc}
\usepackage{palatino}

\raggedright % slipper massa avstavning

% Lägger till § på alla rubriker
\usepackage{titlesec}
\titleformat{\subsection}{\normalfont\large\bfseries}{\S\thesubsection}{0.5em}{}

\usepackage{xfrac}% för 3/4-majoritet....


\newcommand{\change}[3]{\subsection*{#1}
  \subsubsection*{Föreslagna paragrafen:}
  #2
  \subsubsection*{Nuvarande paragraf:}
  #3}

\begin{document}

\section*{Proposition för förändring av FIAs stadgar}
Styrelsen yrkar på att årsmötet antar följande förändring i stadgarna.


\change{§1.2 Obunden}{
Föreningen för intelligenta autonoma system är en religiöst
fackligt och politiskt obunden förening.}{
Föreningen för intelligenta autonoma system är en religiöst, fackligt och partipolitiskt obunden förening.}

\change{§2.3 Avstängning av medlem}{
Styrelsen äger rätt att stänga av medlem från föreningen fram till
nästa årsmöte om denne har skadat föreningens verksamhet eller
anseende. Den avstängningshotade måste få möjlighet att försvara sig
inför styrelsen före beslutet.
Avstängd medlem kan ansöka om nytt medlemsskap vid nästkommande
årsmötet som tar beslut om beviljande.}{
Styrelsen äger rätt att stänga av medlem från föreningen fram till nästa årsmöte om denne har skadat föreningens verksamhet eller anseende. Den avstängningshotade måste få möjlighet att försvara sig inför styrelsen före beslutet. Avstängd medlem måste diskuteras på nästa årsmöte. Antingen så upphävs då avstängningen eller så utesluts medlemmen permanent.}

\change{§2.4 Medlems ansvar}{
Medlem ansvarar för det materiel medlemen använder. Fel som uppkommer
eller upptäcks skall anmälas till styrelsen.

Föreningens materiel får ej användas i kommersiellt syfte. Medlem
är skyldig att följa av styrelsen uppsatta ordningsregler vilka
ska finnas tillgängliga i anslutning till
föreningens materiel.}{
Medlem ansvarar för den materiel han använder. Fel som uppkommer eller upptäcks skall anmälas till styrelsen.

Föreningens materiel får ej användas i kommersiellt syfte. Medlem är skyldig att följa av styrelsen uppsatta ordningsregler. Ordningsreglerna skall finnas tillgängliga i anslutning till föreningens materiel.}

\change{§3.1 Verksamhets- och räkneskapsår}{
Föreningens verksamhets och räkenskapsår löper från och med den 1
augusti till och med den 31 juli.}{
Föreningens verksamhets och räkenskapsår löper från 1 augusti till 31 juli.}

\change{§3.2 Firmatecknare}{
Firmatecknare för föreningen är ordförande och kassör, var för sig,
i den sittande styrelsen.}{
Föreningens firma tecknas av ordförande och kassör var för sig.}

\change{§4.1 Allmänt}{
För att främja föreningens syfte, dvs att öka intresset för robotik
eller artificiell intelligens, uppmanas och uppmuntras medlemmar att
driva relevanta projekt. De kan med fördel drivas som en studiecirkel.
Ett exempel på projekt är en föreläsningserie med föreläsare
från universitetet eller näringslivet.
Det kan även vara att arrangera tävlingar eller workshops inom robotik
och/eller artificiell intelligens (till exempel RoboCup Junior eller
First Lego Leauge).}{
För att främja föreningens syfte, dvs att öka intresset för robotik eller artificiell intelligens, uppmanas och uppmuntras medlemmar att driva relevanta projekt. De kan med fördel drivas som en studiecirkel.
Exempel på projekt skulle kunna vara en föreläsningserie med personer från universitetet eller näringslivet.
Det kan även vara att arrangera tävlingar eller workshops inom robotik eller artificiell intelligens. Till exempel RoboCup Junior eller First Lego Leauge, vilka inte tidigare har arrangerats i Östergötland.}

\change{§4.3 Start av projekt}{
Projekt startas genom att en eller flera medlemmar presenterar sin idé
på ett styrelsemöte. De skall även presentera en budget och tidsplan
för projektet. Styrelsen beslutar därefter om projektet ska godkännas.}{
Projekt startas genom att en eller flera medlemmar presenterar sin idé på ett styrelsemöte. De skall även presentera en budget och tidsplan för projektet. Huruvida projektet skall genomföras bestäms av styrelsen genom votering.}

\change{§4.4 Drift av projekt}{
Projektansvarig är skyldig att meddela styrelsen vid eventuella
ändringar av budget eller tidsplan. Denne bör även rapportera hur
projektet fortlöper. Styrelsen skall vara behjälplig vid kontakt
med universitetet eller andra utomstående. Till exempel för
lokalbokning eller ekonomi.

Projekt har inga egna tillgångar, de förvaltas genom föreningen.
Projektgrupperna är enbart drivande, medan föreningen står som ägare.}{
Projektansvarig är skyldig att meddela styrelsen vid eventuella ändringar av budget eller tidsplan. Denne bör även rapportera hur projektet fortlöper. Styrelsen skall vara behjälplig vid kontakt med universitetet eller andra utomstående, för exempelvis lokalbokning eller ekonomi.

Projekt har inga egna tillgångar, de förvaltas genom föreningen. Projektgrupperna är enbart drivande, medan föreningen står som ägare.}

\change{§4.5 Avslutande av projekt}{
Styrelsen har rätten att avbryta ett projekt genom ett styrelsebeslut.
Projektet kan även avslutas av deltagarna inom projektet.
Föreningen är huvudansvarig för projekt som drivs genom föreningen.}{
Styrelsen har alltid rätten att avbryta ett projekt efter ett majoritetsbeslut inom styrelsen. Projektet kan även avslutas av deltagarna inom projektet.
Föreningen är huvudansvarig för projekten och eventuella tillgångar skall tillfalla föreningen efter avslutade projekt.}


\change{§5.3 Mötets behöriga utlysande}{
\textbf{§5.3 Mötets behöriga sammankallande}
För att årsmötet ska vara behörigt sammankallat ska:
\begin{itemize}
\item En kallelse ha skickats
ut två veckor före mötet via epost. Dagordningen för årsmötet ska
vara bifogad i kallelsen.
\item Hälften eller tio (vilket som är minst) av föreningens medlemar
är närvarande.
\end{itemize}}{
\textbf{§5.3 Mötets behöriga utlysande}
För att vara behörigt utlyst måste föreningens medlemmar meddelas via epost minst två veckor i förväg. Dagordningen för årsmötet skall ingå i kallelsen.}

\change{§5.5 Offentliga dokument}{
Alla dokument som ska redovisas eller diskuteras under ett årsmöte
ska lämnas in till styrelsen senast
7 kalenderdagar före mötets början. Dokumenten, tillsammans med
styrelsen svar i de fallen svar är relevant, ska finnas tillgängliga
för föreningens medlemmar minst tre dagar
före mötets öppnad.
Detta inkluderar, men är inte begränsat till:
\begin{itemize}
\item Dagordning
\item Propositioner
\item Motioner
\item Ekonomisk redovisning
\item Verksamhetsredovisning
\item föregående årsmötesprotokoll
\end{itemize}}{
Alla dokument som skall redovisas eller diskuteras under ett årsmöte skall offentliggöras och lämnas in till styrelsen senast 7 kalenderdagar före mötets början.
Detta inkluderar, men är inte begränsat till:
\begin{itemize}
\item Dagordning
\item Propositioner
\item Motioner
\item Ekonomisk redovisning
\item Verksamhetsredovisning
\end{itemize}}

\change{§5.6 Dagordning vid årsmöte}{
Följande frågor skall behandlas och protokollföras av årsmötet:
\begin{enumerate}
\item Fastställande av röstlängd
\item Val av ordförande
\item Val av sekreterare för årsmötet
\item Val av justerare och tillika rösträknare
\item Beslut om mötet är behörigt sammankallat
\item Godkännande av dagordning
\item Redovisning av föreningens projekt
\item Godkännande av verksamhetsberättelse
\item Godkännande av revisionsberättelse
\item Besulta om ansvarsfrihet för avgående styrelse
\item Val av styrelseordförande
\item Val av styrelsens kassör
\item Val av minst 3 styrelseledamöter
\item Val av relevant antal suppleanter
\item Val av revisor(er)
\item Val av inspektor
\item Val av sammankallande för valberedning
\item Fastställande av medlemsavgifter för kommande verksamhetsår
\item Fastställande av budget för kommande verksamhetsår
\item Övriga frågor
\end{enumerate}}{
Följande frågor skall behandlas och protokollföras av årsmötet:
\begin{enumerate}
\item Fastställande av röstlängd
\item Val av ordförande
\item Val av sekreterare för årsmötet
\item Val av justerare och tillika rösträknare
\item Fråga om mötets behöriga utlysande
\item Godkännande av dagordning
\item Uppläsande av föregående årsmötes{}protokoll% Bugg i TeXworks gör att _ordet_ inte kan sitta ihop för då kraschar det
\item Redovisning av föreningens projekt
\item Godkännande av verksamhetsberättelse
\item Godkännande av revisionsberättelse
\item Fråga om ansvarsfrihet för avgående styrelse
\item Val av styrelseordförande
\item Val av minst 4 styrelseledamöter
\item Val av suppleanter
\item Val av revisor(er)
\item Val av inspektor
\item Val av valberedning
\item Fastställande av medlemsavgifter för kommande verksamhetsår
\item Fastställande av budget för kommande verksamhetsår
\item Övriga frågor
\end{enumerate}}

\change{§6.1 Rätten till extra årsmöte}{
Extra årsmöte kan begäras av:
\begin{itemize}
\item Styrelsen
\item Inspektor
\item Revisor
\item Minst fem medlemmar (eller en fjärdedel av föreningens medlemar)
genom skriftlig begäran till styrelsen
\end{itemize}}{
Extra årsmöte kan begäras av:
\begin{itemize}
\item Styrelsen
\item Inspektor
\item Revisor
\item Minst fem medlemmar som anhåller om detta
\end{itemize}}



\change{§6.2 Kallelse till extra årsmöte}{
Extra årsmöte skall hållas inom 30 kalenderdagar som infaller under
terminstid för den tekniska fackulteten vid Linköpings Universitet efter att
yrkande har inkommit till styrelsen. Årsmötet måste hållas under en av
tekniska fackulteten vid Linköpings Universitets läsperioder.
Kallelse sker enligt §5.
För extra årsmöte gäller samma föreskrifter som för årsmöte,
där det är tilllämpbart.}{
Extra årsmöte skall hållas inom 21 kalenderdagar efter att yrkande därpå inkommit till styrelsen. Kallelse sker enligt §5. För extra årsmöte gäller samma föreskrifter som för årsmöte, i tillämpliga delar.}


\change{§7.2 Mandattid och tillträde}{
Styrelsen tillträder efter ordinarie årsmöte och innehar
sina poster till nästa ordinarie årsmöte om de inte avsätts genom
misstroendeförklaring.}{
Styrelsen tillträder efter ordinarie årsmöte och innehar sina poster till nästa ordinarie årsmöte.}

\change{§7.6 Administration av ekonomi}{
Kassören och styrelsen reglerar föreningens tillgångar i strävan efter
att följa budgeten som fastställdes under föregående ordinarie
årsmötet samt extrainsatta årsmötens beslut.}{
Kassören och styrelsen reglerar föreningens tillgångar i enlighet med den budget som är fastställd under det senaste årsmötet.}


\change{Ny}{
\textbf{§7.7 Misstroende för styrelsen}
Anser föreningens medlemmar att styrelsen inte utför sin uppgift på
ett korrekt sätt ska dessa kräva ett extrainsatt årsmötet.
På det extrainsatta årsmötet krävs att 3/4 av mötet, exklusive
styrelsemedlemmar, röstar för styrelsens avgång.

Efter en misstroendeförklaring ska den sittande styrelsen kalla till
ytterligar ett extrainsatt årsmöte. Till årsmötet ska valberedning ha
tagit fram ett förslag på en styrelse som sitter till och med nästa
ordinarie årsmöte.}{Inga}


\change{§9.2 Begäran om handlingar}{
Revisorerna äger rätt att ta del av samtliga protokoll och övriga handlingar.}{
Revisorerna äger rätt att taga del av samtliga protokoll och övriga handlingar.}

\change{§9.3 Sammanträdesrätt}{
Revisorerna äger rätt att närvara vid föreningens sammanträden.}{
Revisorerna äger rätt att närvara vid samtliga sammanträden vid föreningen.}

\change{§11.1 Ändring av stadgar}{
Beslut om stadgeändring är en tvådelad process där ändringarna
först ska presenteras på ett årsmöte öppet för alla medlemmar.
Vid ett efterföljande årsmöte med minst fyra veckors mellanrum
görs sedan en omröstning om antagandet av de nya stadgarna. Vid
omröstningen för ändring av stadgarna ska det krävas
\(\sfrac{3}{4}\)-majoritet för att beslutet skall antas.}{
Beslut om stadgeändring är en tvådelad process där ändringarna först skall presenteras på ett årsmöte öppet för alla medlemmar.
Vid ett efterföljande årsmöte med minst fyra veckors mellanrum görs sedan en omröstning om antagandet av de nya stadgarna. Vid omröstningen för ändring av stadgarna skall det krävas \(\sfrac{3}{4}\)-majoritet för att beslutet skall antas. Omröstningen måste ske under en av Tekniska Högskolans läsperioder.}

\change{§12.2 Fördelning av tillgångar}{
Beslut om upplösande av föreningen ska innehålla beslut om
ansvarsfrihet för avgående styrelse, samt beslut om hur
föreningens tillgångar ska fördelas för att bäst främja föreningens
syfte.}{
Beslut om upplösande av föreningen skall innehålla beslut om ansvarsfrihet för avgående styrelse, samt beslut om hur föreningens tillgångar skall fördelas.}
\end{document}
