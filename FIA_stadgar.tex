\documentclass[11pt,a4paper]{article}

\usepackage[swedish]{babel}
\usepackage[utf8]{inputenc}
\usepackage[T1]{fontenc}
\usepackage{palatino}

\raggedright % slipper massa avstavning

% Lägger till § på alla rubriker
\usepackage{titlesec}
\titleformat{\section}{\normalfont\Large\bfseries}{\S\thesection}{1em}{}
\titleformat{\subsection}{\normalfont\large\bfseries}{\S\thesubsection}{0.5em}{}

\usepackage{xfrac}% för 3/4-majoritet....

\title{Stadgar för\\Föreningen för intelligenta autonoma system}
\date{14 Sep 2023}

\begin{document}
\maketitle

\section{Föreningen för intelligenta autonoma system}

\subsection{Syfte}
Föreningen för intelligenta autonoma system är en ideell studentförening vid Linköpings Universitet vars mål är att öka intresset för robotik och artificiell intelligens, genom att driva projekt inom dessa områden. 
\subsection{Obunden}
Föreningen för intelligenta autonoma system är en religiöst fackligt och
politiskt obunden förening.
\subsection{Säte}
Föreningen för intelligenta autonoma system har sitt säte i Linköping.

\section{Medlemmar}

\subsection{Vem kan bli medlem?}
Alla studerande vid, anställda vid eller alumner från Linköpings Universitet kan bli medlemmar i Föreningen för intelligenta autonoma system. Medlemskap kan även medges övriga efter styrelsebeslut, men minst 51\% av föreningens medlemmar måste infalla inom de tre ovan nämnda kategorierna. 
\subsection{Medlemsavgift}
Medlemmar skall betala av årsmötet bestämd medlemsavgift.
\subsection{Avstängning av medlem}
Styrelsen äger rätt att stänga av medlem från föreningen fram till nästa årsmöte om denne har skadat föreningens verksamhet eller anseende. Den avstängningshotade måste få möjlighet att försvara sig inför styrelsen före beslutet. Avstängd medlem kan ansöka om nytt medlemskap vid nästkommande årsmöte, som tar beslut om beviljande.
\subsection{Medlems ansvar}
Medlem ansvarar för det materiel medlemmen använder. Fel som uppkommer eller upptäcks skall anmälas till styrelsen. Föreningens materiel får ej användas i kommersiellt syfte. Medlem är skyldig att följa av styrelsen uppsatta ordningsregler vilka ska finnas
tillgängliga i anslutning till föreningens materiel.
\subsection{Förnyande av medlemskap }
\label{sec:förnyande}
Medlemskap ska förnyas en gång per kalenderår. Medlemmen ska förfrågas om denne önskar bevara sitt medlemskap och ges minst en vecka att svara på detta. Även avsaknaden av ett svar ska tolkas som en vilja att uppsäga sitt medlemskap. Det är belagt styrelsen att genomföra detta för samtliga medlemmar varje verksamhetsår.
\subsection{Medlemsinformation}
Medlem ska vid start av medlemskap uppge nedanstående information. Medlem ska även ges möjlighet att uppdatera dessa genom åtminstone den årliga medlemskapsbekräftelsen, se §\ref{sec:förnyande}. \bigskip
Informationen som behöver uppges är följande:
\begin{itemize}
    \item För- och efternamn.
    \item E-post.
    \item Födelsedatum.
    \item Studestatus. En av:
    \begin{itemize}
        \item Student vid LiU.
        \item Alumn från LiU.
        \item Anställd vid LiU.
        \item Ej anknuten till LiU. 
    \end{itemize}
\end{itemize}

\section{Organisation}

\subsection{Verksamhets- och räkenskapsår}
Föreningens verksamhets- och räkenskapsår löper från och med den 1
augusti till och med den 31 juli.
\subsection{Firmatecknare}
Firmatecknare för föreningen är ordförande och kassör, var för sig, i den
sittande styrelsen. 
\subsection{Verksamheten och beslutande organ}
Styrelsen är skyldig att agera för föreningens bästa. Verksamheten utövas av:
\begin{enumerate}
\item Årsmöten
\item Styrelsen
\item Tävlings- och Projektansvariga
\end{enumerate}


\section{Projekt}

\subsection{Allmänt}
För att främja föreningens syfte, dvs att öka intresset för robotik eller
artificiell intelligens, uppmanas och uppmuntras medlemmar att driva
relevanta projekt. De kan med fördel drivas som en studiecirkel. Ett
exempel på projekt är en föreläsningserie med föreläsare från universitetet
eller näringslivet. Det kan även vara att arrangera tävlingar eller
workshops inom robotik och/eller artificiell intelligens (till exempel
RoboCup Junior eller First Lego Leauge).

\subsection{Ansvarig medlem}
Projekt som drivs av föreningen skall ha en, av styrelsen vald, ansvarig medlem.

\subsection{Start av projekt}
Projekt startas genom att en eller flera medlemmar presenterar sin idé på
ett styrelsemöte. De skall även presentera en budget och tidsplan för
projektet. Styrelsen beslutar därefter om projektet ska godkännas.

\subsection{Drift av projekt}
Projektansvarig är skyldig att meddela styrelsen vid eventuella ändringar
av budget eller tidsplan. Denne bör även rapportera hur projektet
fortlöper. Styrelsen skall vara behjälplig vid kontakt med universitetet
eller andra utomstående. Till exempel för lokalbokning eller ekonomi.
Projekt har inga egna tillgångar, de förvaltas genom föreningen.
Projektgrupperna är enbart drivande, medan föreningen står som ägare.

\subsection{Avslutande av projekt}
Styrelsen har rätten att avbryta ett projekt genom ett styrelsebeslut.
Projektet kan även avslutas av deltagarna inom projektet. Föreningen är
huvudansvarig för projekt som drivs genom föreningen.



\section{Ordinarie årsmöte}

\subsection{Högsta beslutande organ}
Ordinarie årsmöte är föreningens högsta beslutande organ.
\subsection{Tid för ordinarie årsmöte}
Ordinarie årsmötet skall hållas i början av höstens första läsperiod. 
\subsection{Mötets behöriga sammankallande}
För att årsmötet ska vara behörigt sammankallat ska:
\begin{itemize}
\item En kallelse ha skickats ut till samtliga medlemmar två veckor före mötet via e-post. Dagordningen för årsmötet ska vara bifogad i kallelsen. 
\item Hälften eller tio (vilket som är minst) av föreningens medlemar är
närvarande.
\end{itemize}
\subsection{Rösträtt}
Endast närvarande medlem har rösträtt på årsmöte.
\subsection{Offentliga dokument}
Alla dokument som ska redovisas eller diskuteras under ett årsmöte ska
lämnas in till styrelsen senast sju kalenderdagar före mötets början.
Dokumenten, tillsammans med styrelsen svar i de fallen svar är relevant,
ska finnas tillgängliga för föreningens medlemmar minst tre dagar före
mötets öppnad. Detta inkluderar, men är inte begränsat till:
\begin{itemize}
\item Dagordning
\item Propositioner
\item Motioner
\item Ekonomisk redovisning
\item Verksamhetsredovisning
\item Föregående årsmötesprotokoll
\end{itemize}
\subsection{Dagordning vid årsmöte}
Följande frågor skall behandlas och protokollföras av årsmötet:
\begin{enumerate}
\item Fastställande av röstlängd
\item Val av ordförande
\item Val av sekreterare för årsmötet
\item Val av justerare och tillika rösträknare
\item Beslut om mötet är behörigt sammankallat
\item Godkännande av dagordning
\item Redovisning av föreningens projekt
\item Godkännande av verksamhetsberättelse
\item Godkännande av revisionsberättelse
\item Redovisning av föreningens ekonomi
\item Beslut om ansvarsfrihet för avgående styrelse
\item Val av styrelseordförande
\item Val av sytelsens kassör
\item Val av minst 3 styrelseledamöter
\item Val av relevant antal suppleanter
\item Val av revisor(er)
\item Val av inspektor
\item Val av valberedning
\item Fastställande av medlemsavgifter för kommande verksamhetsår
\item Fastställande av budget för kommande verksamhetsår
\item Övriga frågor
\end{enumerate}
\subsection{Beslutsform}
Beslut i fråga som upptagits på dagordningen fattas med enkel majoritet. I
ärenden upptagna under övriga frågor fordras \(\sfrac{3}{4}\)-majoritet. För beslut om stadgeändring gäller särskilda regler, se \S11
\subsection{Övriga frågor}
Under övriga frågor får ej ekonomiska beslut fattas.




\section{Extra årsmöte}

\subsection{Rätten till extra årsmöte}
Extra årsmöte kan begäras av:
\begin{itemize}
\item Styrelsen
\item Inspektor
\item Revisor
\item Minst fem medlemmar eller en fjärdedel av föreningens medlemmar,
vilket som är minst, genom skriftlig begäran till styrelsen
\end{itemize}
\subsection{Kallelse till extra årsmöte}
Extra årsmöte skall hållas inom 30 kalenderdagar som infaller under
terminstid för den tekniska fakulteten vid Linköpings Universitet efter
att yrkande har inkommit till styrelsen. Årsmötet måste hållas under en av
tekniska fakulteten vid Linköpings Universitets läsperioder. Kallelse sker
enligt \S5. För extra årsmöte gäller samma föreskrifter som för årsmöte, där
det är tilllämpbart.



\section{Styrelse}

\subsection{Poster}
Styrelsen skall innefatta följande poster:
\begin{itemize}
\item Ordförande
\item Kassör
\item Sekreterare
\item IT-ansvarig
\item Ledamöter
\end{itemize}
\subsection{Mandattid och tillträde}
Styrelsen tillträder efter ordinarie årsmöte och innehar sina poster till
nästa ordinarie årsmöte om de inte väljer att avgå eller avsätts genom
misstroendeförklaring.
\subsection{Beslutsmässighet}
Styrelsen är beslutsmässig då minst hälften av styrelsen är närvarande.
\subsection{Protokoll}
Alla styrelsemöten skall protokollföras samt arkiveras. 
\subsection{Styrelsens ansvar}
Styrelsen är ansvarig för  föreningens verksamhet samt ekonomi och är ansvarig inför årsmötet.
\subsection{Administration av ekonomi}
Kassören och styrelsen reglerar föreningens tillgångar i strävan efter att
följa budgeten som fastställdes under föregående ordinarie årsmötet samt
extrainsatta årsmötens beslut.
\subsection{Misstroende för styrelsen}
Anser föreningens medlemmar att
styrelsen inte utför sin uppgift på ett korrekt sätt ska dessa kräva ett
extrainsatt årsmötet. På det extrainsatta årsmötet krävs att 3/4 av mötet,
exklusive styrelsemedlemmar, röstar för styrelsens avgång.
Efter en misstroendeförklaring ska den sittande styrelsen kalla till
ytterligar ett extrainsatt årsmöte. Till årsmötet ska valberedning ha tagit
fram ett förslag för val av ny styrelse. Den nyvalda
styrelsen sitter till och med nästa ordinarie årsmöte.
\subsection{Entledigande från styrelsen}
Styrelsemedlemmar eller suppleanter kan entledigas av styrelsen om hen önskar avgå. Om
en av firmatecknarna önskar avgå krävs dock ett extra årsmöte för att
välja ny firmatecknare. Beslut om ansvarsfrihet beviljas som vanligt av
årsmötet, men styrelseledamoten kan endast hållas ansvarig för beslut
som fattats innan entledigandet.
\subsection{Ekonomiska beslut}
Ekonomiska beslut som fattas av styrelsen skall tas under punkter
inskrivna på dagordningen och ej under övriga frågor.

\section{Inspektor}

\subsection{Val av inspektor}
Föreningens inspektor väljs under årsmötet.
\subsection{Uppgift}
Inspektorn har som uppgift att övervaka föreningens verksamhet och kontrollera att denna sker på ett adekvat vis.\subsection{Medlemskap}
Inspektorn behöver ej vara medlem i föreningen, men måste vara anställd vid Linköpings universitet.
\subsection{Närvarorätt}
Inspektorn har närvaro- och yttranderätt på styrelsesammanträden.


\section{Revision}

\subsection{Val av revisor}
Vid årsmötet ska minst en revisor väljas för det kommande verksamhetsåret. Dessa får ej inneha någon annan befattning inom föreningen.
\subsection{Begäran om handlingar}
Revisorerna äger rätt att ta del av samtliga protokoll och övriga handlingar.
\subsection{Sammanträdesrätt}
Revisorerna äger rätt att närvara vid föreningens sammanträden.
\subsection{Revisionsberättelse}
Revisorerna skall efter verkställd granskning upprätta en revisionsberättelse i vilken de av- eller tillstyrker ansvarsfrihet för styrelsen. Revisionsberättelsen skall innan den föredrages på årsmötet ha varit tillgänglig för medlemmarna i minst 7 dagar. Revisionsberättelsen skall inlämnas och publiceras precis som andra handlingar inför årsmötet.



\section{Tolkning av stadgar}

\subsection{Oklarheter}
Om det uppstår oklarheter över tolkningen av föreningens stadgar gäller styrelsens tolkning fram tills nästa årsmöte.



\section{Stadgeändring}

\subsection{Ändring av stadgar}
Beslut om stadgeändring är en tvådelad process där ändringarna först ska
presenteras på ett årsmöte öppet för alla medlemmar. Vid ett efterföljande
årsmöte med minst fyra veckors mellanrum görs sedan en omröstning om
antagandet av de nya stadgarna. Vid omröstningen för ändring av stadgarna 
ska det krävas \(\sfrac{3}{4}\)-majoritet för att beslutet skall antas.



\section{Upplösande}

\subsection{Upplösning}
Föreningen kan upplösas enligt de regler som gäller för stadgeändring.
\subsection{Fördelning av tillgångar}
Beslut om upplösande av föreningen ska innehålla beslut om
ansvarsfrihet för avgående styrelse, samt beslut om hur föreningens
tillgångar ska fördelas för att bäst främja föreningens syfte.



\end{document}

\grid
