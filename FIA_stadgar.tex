\documentclass[11pt,a4paper]{article}

\usepackage[swedish]{babel}
\usepackage[utf8]{inputenc}
\usepackage[T1]{fontenc}
\usepackage{palatino}

\raggedright % slipper massa avstavning

% Lägger till § på alla rubriker
\usepackage{titlesec}
\titleformat{\section}{\normalfont\Large\bfseries}{\S\thesection}{1em}{}
\titleformat{\subsection}{\normalfont\large\bfseries}{\S\thesubsection}{0.5em}{}

\usepackage{xfrac}% för 3/4-majoritet....

\title{Stadgar för\\Föreningen för intelligenta autonoma system}
\date{23 Maj 2013}

\begin{document}
\maketitle

\section{Föreningen för intelligenta autonoma system}

\subsection{Syfte}
Föreningen för intelligenta autonoma system är en ideell studentförening vid Linköpings Universitet vars mål är att öka intresset för robotik och artificiell intelligens, genom att driva projekt inom dessa områden. 
\subsection{Obunden}
Föreningen för intelligenta autonoma system är en religiöst, fackligt och partipolitiskt obunden förening.
\subsection{Säte}
Föreningen för intelligenta autonoma system har sitt säte i Linköping.

\section{Medlemmar}

\subsection{Vem kan bli medlem?}
Alla studerande eller anställda vid Linköpings universitet kan bli medlemmar i föreningen för intelligenta autonoma system.
Medlemskap kan även medges utomstående efter styrelsebeslut. Dock måste minst 51\% av medlemmarna vara studenter från Linköpings Universitet. 
\subsection{Medlemsavgift}
Medlemmar skall betala av årsmötet bestämd medlemsavgift.
\subsection{Avstängning av medlem}
Styrelsen äger rätt att stänga av medlem från föreningen fram till nästa årsmöte om denne har skadat föreningens verksamhet eller anseende. Den avstängningshotade måste få möjlighet att försvara sig inför styrelsen före beslutet. Avstängd medlem måste diskuteras på nästa årsmöte. Antingen så upphävs då avstängningen eller så utesluts medlemmen permanent.
\subsection{Medlems ansvar}
Medlem ansvarar för den materiel han använder. Fel som uppkommer eller upptäcks skall anmälas till styrelsen.
Allt arbete sker på egen risk.
Föreningens materiel får ej användas i kommersiellt syfte. Medlem är skyldig att följa av styrelsen uppsatta ordningsregler. Ordningsreglerna skall finnas tillgängliga i anslutning till föreningens materiel.



\section{Organisation}

\subsection{Verksamhets- och räkenskapsår}
Föreningens verksamhets och räkenskapsår löper från 1 augusti till 31 juli.
\subsection{Firmatecknare}
Föreningens firma tecknas av ordförande och kassör var för sig. 
\subsection{Verksamheten och beslutande organ}
Styrelsen är skyldig att agera för föreningens bästa. Verksamheten utövas av:
\begin{enumerate}
\item Årsmöten
\item Styrelsen
\item Tävlings- och Projektansvariga
\end{enumerate}


\section{Projekt}

\subsection{Allmänt}
För att främja föreningens syfte, dvs att öka intresset för robotik eller artificiell intelligens, uppmanas och uppmuntras medlemmar att driva relevanta projekt. De kan med fördel drivas som en studiecirkel. 
Exempel på projekt skulle kunna vara en föreläsningserie med personer från universitetet eller näringslivet.
Det kan även vara att arrangera tävlingar eller workshops inom robotik eller artificiell intelligens. Till exempel RoboCup Junior eller First Lego Leauge, vilka inte tidigare har arrangerats i Östergötland. 

\subsection{Ansvarig medlem}
Projekt som drivs av föreningen skall ha en, av styrelsen vald, ansvarig medlem.

\subsection{Start av projekt}
Projekt startas genom att en eller flera medlemmar presenterar sin idé på ett styrelsemöte. De skall även presentera en budget och tidsplan för projektet. Huruvida projektet skall genomföras bestäms av styrelsen genom votering.

\subsection{Drift av projekt}
Projektansvarig är skyldig att meddela styrelsen vid eventuella ändringar av budget eller tidsplan. Denne bör även rapportera hur projektet fortlöper. Styrelsen skall vara behjälplig vid kontakt med universitetet eller andra utomstående, för exempelvis lokalbokning eller ekonomi.

Projekt har inga egna tillgångar, de förvaltas genom föreningen. Projektgrupperna är enbart drivande, medan föreningen står som ägare. 

\subsection{Avslutande av projekt}
Styrelsen har alltid rätten att avbryta ett projekt efter ett majoritetsbeslut inom styrelsen. Projektet kan även avslutas av deltagarna inom projektet. 
Föreningen är huvudansvarig för projekten och eventuella tillgångar skall tillfalla föreningen efter avslutade projekt. 




\section{Ordinarie årsmöte}

\subsection{Högsta beslutande organ}
Ordinarie årsmöte är föreningens högsta beslutande organ.
\subsection{Tid för ordinarie årsmöte}
Ordinarie årsmötet skall hållas i början av höstens första läsperiod. 
\subsection{Mötets behöriga utlysande}
För att vara behörigt utlyst måste föreningens medlemmar meddelas via epost minst två veckor i förväg. Dagordningen för årsmötet skall ingå i kallelsen.
\subsection{Rösträtt}
Endast närvarande medlem har rösträtt på årsmöte.
\subsection{Offentliga dokument}
Alla dokument som skall redovisas eller diskuteras under ett årsmöte skall offentliggöras och lämnas in till styrelsen senast 7 kalenderdagar före mötets början.
Detta inkluderar, men är inte begränsat till:
\begin{itemize}
\item Dagordning
\item Propositioner
\item Motioner
\item Ekonomisk redovisning
\item Verksamhetsredovisning
\end{itemize}
\subsection{Dagordning vid årsmöte}
Följande frågor skall behandlas och protokollföras av årsmötet:
\begin{enumerate}
\item Fastställande av röstlängd
\item Val av ordförande
\item Val av sekreterare för årsmötet
\item Val av justerare och tillika rösträknare
\item Fråga om mötets behöriga utlysande
\item Godkännande av dagordning
\item Uppläsande av föregående årsmötes{}protokoll% Bugg i TeXworks gör att _ordet_ inte kan sitta ihop för då kraschar det
\item Redovisning av föreningens projekt
\item Godkännande av verksamhetsberättelse
\item Godkännande av revisionsberättelse
\item Fråga om ansvarsfrihet för avgående styrelse
\item Val av styrelseordförande
\item Val av minst 4 styrelseledamöter
\item Val av suppleanter
\item Val av revisor(er)
\item Val av inspektor
\item Val av valberedning
\item Fastställande av medlemsavgifter för kommande verksamhetsår
\item Fastställande av budget för kommande verksamhetsår
\item Övriga frågor
\end{enumerate}
\subsection{Beslutsform}
Beslut i fråga som upptagits på dagordningen fattas med enkel majoritet. I ärenden upptagna under övriga frågor fordras \(\sfrac{3}{4}\)-majoritet.





\section{Extra årsmöte}

\subsection{Rätten till extra årsmöte}
Extra årsmöte kan begäras av:
\begin{itemize}
\item Styrelsen
\item Inspektor
\item Revisor
\item Minst fem medlemmar som anhåller om detta
\end{itemize}
\subsection{Kallelse till extra årsmöte}
Extra årsmöte skall hållas inom 21 kalenderdagar efter att yrkande därpå inkommit till styrelsen. Kallelse sker enligt §5. För extra årsmöte gäller samma föreskrifter som för årsmöte, i tillämpliga delar.



\section{Styrelse}

\subsection{Poster}
Styrelsen skall innefatta följande poster:
\begin{itemize}
\item Ordförande
\item Kassör
\item Sekreterare
\item IT-ansvarig
\item Ledamöter
\end{itemize}
\subsection{Mandattid och tillträde}
Styrelsen tillträder efter ordinarie årsmöte och innehar sina poster till nästa ordinarie årsmöte.
\subsection{Beslutsmässighet}
Styrelsen är beslutsmässig då minst hälften av styrelsen är närvarande.
\subsection{Protokoll}
Alla styrelsemöten skall protokollföras samt arkiveras. 
\subsection{Styrelsens ansvar}
Styrelsen är ansvarig för  föreningens verksamhet samt ekonomi och är ansvarig inför årsmötet.
\subsection{Administration av ekonomi}
Kassören och styrelsen reglerar föreningens tillgångar i enlighet med den budget som är fastställd under det senaste årsmötet.




\section{Inspektor}

\subsection{Val av inspektor}
Föreningens inspektor väljs under årsmötet.
\subsection{Uppgift}
Inspektorn har som uppgift att övervaka föreningens verksamhet och kontrollera att denna sker på ett adekvat vis.\subsection{Medlemskap}
Inspektorn behöver ej vara medlem i föreningen, men måste vara anställd vid Linköpings universitet.
\subsection{Närvarorätt}
Inspektorn har närvaro- och yttranderätt på styrelsesammanträden.


\section{Revision}

\subsection{Val av revisor}
Vid årsmötet ska minst en revisor väljas för det kommande verksamhetsåret. Dessa får ej inneha någon annan befattning inom föreningen.
\subsection{Begäran om handlingar}
Revisorerna äger rätt att taga del av samtliga protokoll och övriga handlingar.
\subsection{Sammanträdesrätt}
Revisorerna äger rätt att närvara vid samtliga sammanträden vid föreningen.
\subsection{Revisionsberättelse}
Revisorerna skall efter verkställd granskning upprätta en revisionsberättelse i vilken de av- eller tillstyrker ansvarsfrihet för styrelsen. Revisionsberättelsen skall innan den föredrages på årsmötet ha varit tillgänglig för medlemmarna i minst 7 dagar. Revisionsberättelsen skall inlämnas och publiceras precis som andra handlingar inför årsmötet.



\section{Tolkning av stadgar}

\subsection{Oklarheter}
Om det uppstår oklarheter över tolkningen av föreningens stadgar gäller styrelsens tolkning fram tills nästa årsmöte.



\section{Stadgeändring}

\subsection{Ändring av stadgar}
Beslut om stadgeändring är en tvådelad process där ändringarna först skall presenteras på ett årsmöte öppet för alla medlemmar. 
Vid ett efterföljande årsmöte med minst fyra veckors mellanrum görs sedan en omröstning om antagandet av de nya stadgarna. Vid omröstningen för ändring av stadgarna skall det krävas \(\sfrac{3}{4}\)-majoritet för att beslutet skall antas. Omröstningen måste ske under en av Tekniska Högskolans läsperioder.



\section{Upplösande}

\subsection{Upplösning}
Föreningen kan upplösas enligt de regler som gäller för stadgeändring.
\subsection{Fördelning av tillgångar}
Beslut om upplösande av föreningen skall innehålla beslut om ansvarsfrihet för avgående styrelse, samt beslut om hur föreningens tillgångar skall fördelas.



\end{document}

